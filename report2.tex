\documentclass{ctexart}

\author{李约瀚 \\ 14130140331 \\ qinka@live.com \\ qinka@qinka.pw}
\title{FPGA设计基础实验报告 (二)}

\usepackage{listings}

%\lstset{frame=single,breaklines,numbers}

\begin{document}
    
        % Cover
        \thispagestyle{empty}
        \begin{center}
            \vspace*{4em}
            {\Huge\textbf{FPGA设计基础实验报告\\\vspace*{0.5em} (二)}}
            \vfill
            \begin{tabular}{c@{:}l}
                班级 & 1413014 \\
                学号 & 14130140331 \\ 
                姓名 & 李约瀚 \\ 
                教师 & 沈沛意 \\
            \end{tabular} 
            \vspace*{4em}\\
        \end{center}
        \newpage
        
       
        % Setting
        \setcounter{page}{0}
        \setcounter{section}{0}
        %\renewcommand\thesection{实验编号 1-\numeric{section} 题目: }
        %\renewcommand\thesubsection{}
        %\renewcommand\thesubsubsection{(\numeric{subsubsection})}

        %% Exp 2-1
        
        \section{数字时钟管理实验}
        
        \subsection{实验目的}
        \begin{itemize}
        \item 熟悉ISE软件,会使用ISE软件进行设计和仿真
        \item 学会Xilinx IP核生成工具的使用
        \item 学会数字时钟管理IP核的配置和使用
        \end{itemize}

        \subsection{实验步骤}

        \begin{itemize}
        \item 创建工程
        \item 设计输入
        \item 设计综合
        \item 设计仿真
        \end{itemize}

        \subsection{报告正文}

        \subsubsection{实验原理}

        IP核,也就是知识产权核(\verb|Intellectual Property|),
        是那些己验证的、可重利用的、具有某种确定功能的IC模块。
        分为软核(\verb|soft IP core|)、固核(\verb|firm IP core|)
        和硬核(\verb|hard IP core|)

        IP核将一些在数字电路中常用,但比较复杂的功能块,
        如FIR滤波器、SDRAM控制器、PCI接口等设计成可修改参数的模块,
        在设计可以直接调用,从而大大提高了设计的可重用性,
        在数字系统设计中得到了广泛的应用。

        目前,大型设计一般推荐使用同步时序电路。同步时序电路基于时钟触发沿设计,
        对时钟的周期、占空比、延时和抖动提出了更高的要求。
        为了满足同步时序设计的要求,一般在FPGA设计中采用全局时钟资源
        驱动设计的主时钟,以达到最低的时钟抖动和延迟。
       
        Xilinx在其FPGA系列芯片中都提供了高性能数字时钟管理器(DCM)模块,
        是管理和掌控时钟的专用模块,可以实现时钟的分频,倍频,去抖动,相移等操作。

        \paragraph{新建工程}

        打开 ISE 14.7之后,选择新建工程,将工程的路径设置好。
        在单击下一步之后,选择 Nexys3 的 Spartan6 XC6SLX16-CS324 芯片对应的配置。硬件描述语言选择 Verilog,其中的设计硬件用的描述语言是 Verilog。
        然后点击下一步之后进入工程信息页面并确认无误之后,点击完成结束工程的创建。

        \paragraph{设计输入}

        在菜单 \verb|Project| 中选择 \verb|New Source| 创建新的设计,并选择 \verb|Verilog Module| 创建文件。

        其中输入的设计文件是:
        \begin{lstlisting}[language=Verilog]
`timescale 1ns / 1ps
// Copyright (C) Xilinx
module top(
    input clk,
    input reset,
    output clk_out1,
    output clk_out2,
    output locked
    );
	 
dcm_core dcm0
   (// Clock in ports
    .CLK_IN1(clk),      // IN
    // Clock out ports
    .CLK_OUT1(clk_out1),     // OUT
    .CLK_OUT2(clk_out2),     // OUT
    // Status and control signals
    .RESET(reset),// IN
    .LOCKED(locked));      // OUT

endmodule
        \end{lstlisting}

        添加 IP 核,在菜单 \verb|Project| 中选择 \verb|New Source| 创建新的设计,并选择 \verb|IP (CORE Generator & Architecture Wizard)| 创建IP核。
        然后再\verb|FPGA Features and Design| 中找到 \verb|Clocking Wizard| 创建时钟。在配置页面中将输入频率设置为100MHz,然后在第二个页面中设置两个时钟。频率分别是50MHz 与 200MHz。第二个时钟的相位是180度。

        \paragraph{综合设计}

        在左侧的 \verb|Design| 面板中下方中 双击 \verb|Synthesize-XST| 开始综合过程。

        在左侧的 \verb|Design| 面板中的 \verb|Hierarchy| 中选中创建的 HDL 源文件,在下方的双击 \verb|View RTL Schematic| 查看电路。

        \paragraph{设计仿真}
        
        为工程文件添加仿真文件。在菜单 \verb|Project| 中选择 \verb|New Source| 创建新的设计,并选择 \verb|Verilog Test Fixture| 创键测试文件。

        在测试文件中添加 \lstinline|always #5 clk = ~clk;|
        \begin{lstlisting}
`timescale 1ns / 1ps

module test_bench;

	// Inputs
	reg clk;
	reg reset;

	// Outputs
	wire clk_out1;
	wire clk_out2;
	wire locked;

	// Instantiate the Unit Under Test (UUT)
	top uut (
		.clk(clk), 
		.reset(reset), 
		.clk_out1(clk_out1), 
		.clk_out2(clk_out2), 
		.locked(locked)
	);

	initial begin
		// Initialize Inputs
		clk = 0;
		reset = 0;

		// Wait 100 ns for global reset to finish
		#100;
 
		// Add stimulus here

	end
	
	always #5 clk = ~clk;
      
endmodule
        \end{lstlisting}

        然后在左边面板中选中\verb|Simulation|,在工程管理区选中测试代码,
        然后在过程管理区双击\verb|Simulate Behavioral Model|
        ,\verb|ISE|将启动\verb|ISE Simulator|,运行仿真。

        \paragraph{结果}

        