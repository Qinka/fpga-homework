\documentclass{ctexart}

\usepackage{listings}

\title{FPGA 作业 (II)}
\author{李约瀚 \\ 14130140331 \\ qinka@live.com \\ qinka@qinka.pw}

\lstset{breaklines}

\begin{document}

% Cover
\thispagestyle{empty}
\begin{center}
  \vspace*{4em}
  {\Huge\textbf{FPGA 作业 \\ \vspace*{0.5em} (II)}}
  \vfill
  \large
  \begin{tabular}{c@{:}l}
    班级 & 14113014 \\
    学号 & 14130140331 \\
    姓名 & 李约瀚 \\
    教师 & 沈沛意 \\
  \end{tabular}
  \vspace*{4em}\\
\end{center}
\newpage

\section{层次化工程创建}

\subsection{实验目的}
\begin{itemize}
\item 熟悉简单逻辑门的RTL描述
\item 创建简单电路的结构化描述
\item 用 VHDL 创建层次结构描述
\item 熟悉 ISE 集成环境中的 HDL 编辑器
\end{itemize}


\subsection{实验步骤}

\begin{enumerate}
\item 使用 ISE 创建新的工程
\item 完成逻辑门的 RTL 描述
\item 检查代码中的语法错误并生成原理图
\end{enumerate}


\subsection{报告正文}

\subsubsection{创建一个新工程}

打开 \verb|Xilinx ISE| 在文件中选择创建新的工程项目,并配置相对应的路径。
点击下一步,然后配置与 Spartan-6 和开发版 Nexys3 有关的配置,然后并继续配置。

\subsubsection{逻辑门的 RTL 描述}

在工程中选择新建源码,并将其命名为 \verb|MY\_AND2.vhd|。然后为其输入代码:
\begin{lstlisting}[language=VHDL]
library IEEE;
use IEEE.STD_LOGIC_1164.ALL;

entity my_and2 is
    Port ( A : in  STD_LOGIC;
           B : in  STD_LOGIC;
           C : out  STD_LOGIC);
end my_and2;

architecture Behavioral of my_and2 is
begin
	C <= A and B;
end Behavioral;
\end{lstlisting}

并通过同样的方式创建文件 \verb|MY_OR2.vhd|。然后输入代码:
\begin{lstlisting}[language=VHDL]
library IEEE;
use IEEE.STD_LOGIC_1164.ALL;

entity my_or2 is
    Port ( A : in  STD_LOGIC;
           B : in  STD_LOGIC;
           C : out  STD_LOGIC);
end my_or2;

architecture Behavioral of my_or2 is
begin
	C <= A or B;
end Behavioral;
\end{lstlisting}

然后是文件 \verb|AND_OR.vhd|。并输入代码:
\begin{lstlisting}[language=VHDL]
library IEEE;
use IEEE.STD_LOGIC_1164.ALL;

entity and_or is
  Port ( INP: in STD_LOGIC_VECTOR(3 downto 0);
           Z: out STD_LOGIC);
end and_or;

architecture Struct of and_or is
  component my_and2
    port( A: in  STD_LOGIC;
          B: in  STD_LOGIC;
          C: out STD_LOGIC);
  end component;
  component my_or2
    port( A: in  STD_LOGIC;
          B: in  STD_LOGIC;
          C: out STD_LOGIC);
  end component;

signal s1,s2:STD_LOGIC;
begin
  U0: my_and2 port map (A => INP(0), B => INP(1), C => s1);
  U1: my_and2 port map (A => INP(2), B => INP(3), C => s2);
  U2: my_or2  port map (A => S1    , B => S2    , C => Z );
end struct;
\end{lstlisting}

\subsubsection{语法检查并生成原理图}

在 ISE 中选中顶层模块 AND\_OR,在 Proccess 中进行综合,在 Synthesize 中双击 
\verb|Check Syntax| 检查语法。并修正有错误的语法。
然后在选中 \verb|View RTL Schematic| 生成并查看原理图。


\subsection{额外步骤}

在配置检查原理图之后,为项目添加 约束文件,代码如下:
\begin{lstlisting}

\end{lstlisting}
NET "INP[0]" LOC = T10;
NET "INP[1]" LOC = T9;
NET "INP[2]" LOC = V9;
NET "INP[3]" LOC = M8;
NET "Z" LOC = U16;

NET "INP[3]" IOSTANDARD = LVCMOS33;
NET "INP[2]" IOSTANDARD = LVCMOS33;
NET "INP[1]" IOSTANDARD = LVCMOS33;
NET "INP[0]" IOSTANDARD = LVCMOS33;
NET "Z" IOSTANDARD = LVCMOS33;
\end{document}

将输入与波动开关绑定,然后输出绑定到LED灯。
然后进行综合生成比特流文件,并配置到开发板上。拨动开关,然后观察结果。